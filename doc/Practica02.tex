\documentclass[letterpaper,12pt]{article}
% Soporte para los acentos.
\usepackage[utf8]{inputenc}
\usepackage[T1]{fontenc}
% Idioma español.
\usepackage[spanish,mexico, es-tabla]{babel}
% Soporte de símbolos adicionales (matemáticas)
\usepackage{multirow}
\usepackage{amsmath}
\usepackage{amssymb}
\usepackage{amsthm}
\usepackage{amsfonts}
\usepackage{latexsym}
\usepackage{enumerate}
\usepackage{ragged2e}
\usepackage{mathtools}

% Soporte para referencias y citas
\usepackage{hyperref}

% Soporte para imágenes.
\usepackage{graphicx}
% Modificamos los márgenes del documento.
\usepackage[lmargin=2cm,rmargin=2cm,top=2cm,bottom=2cm]{geometry}

\title{Fundamentos de Bases de Datos \\
       Práctica 02 \\
       Manipulación de archivos}
       
\author{Teresa Becerril Torres
        $\#$ de cuenta: $315045132$ \\
        Miguel Ángel Torres Sánchez
        $\#$ de cuenta: $315300442$ \\
        Nicole Romina Traschikoff García
        $\#$ de cuenta: $315164482$ \\
        Tania Michelle Rubí Rojas
        $\#$ de cuenta: $315121719$}

\date{02 de Septiembre del 2019}

\begin{document}
\maketitle

\section{Análisis de requerimientos.}

Se necesita crear un prototipo de un sistema que capture la información del 
personal del transporte de la Ciudad de México. La información capturada se 
administrará en archvivos $.cvs$ y será utilizada posteriormente para poblar 
una base de datos. El sistema debe:

\begin{itemize}
    \item Capturar información.
	\item Consultar información.
	\item Editar información.
	\item Eliminar información.
	\item Guardar y leer información en archivos .cvs .
\end{itemize}

Los actores del sistema son los empleados, los vehículos, las licencias y los 
exámenes médicos.

Un empleado contiene su información básica: nombre completo, fecha de 
nacimiento, dirección, sexo, grado máximo de estudios y el horario en el que 
labora. Cada empleado conduce un vehículo, tiene una licencia de conducir 
válida, así como un registro de sus licencias pasadas vencidas, y un examen 
médico hecho hace no más de 6 meses.

Los vehículos tienen un identificador único, saben el tipo de combustible que 
necesitan y su capacidad de pasajeros. Un vehículo es operado por los 
empleados y puede ser operado por más de uno.

Una licencia contiene un identificador único, su fecha de expedición, su fecha 
de vencimiento y su tipo de licencia. Las licencias pertenecen a un solo 
empleado y pueden ser renovadas cuando llegan a su fecha de vencimiento.

El examen médico pertenece a un solo empleado y contiene información sobre 
su peso, talla, presión y estatus. Además mantiene un registro de fecha y 
hora en que se realizó el examen al empleado y la cédula del médico encargado.

\section{Actividad.}

\begin{enumerate}
    \item Incluir cinco diferencias entre almacenar la información 
    utilizando un sistema de archivos a almacenarla utilizando una base de 
    datos. Describe que es más conveniente utilizar.

    \textsc{Solución:} Es más conveniente utilizar una Base de Datos ya que
    ésta nos proporciona más herramientas para manejar nuestros archivos, 
    además de que éstos se encuentran más seguros en caso de que un error 
    externo llegase a ocurrir. 

    
    \begin{itemize}
        \item En el sistema de archivos es muy difícil lograr el acceso 
        concurrente a la información; mientras que en las bases de datos 
        es fácilmente realizable esta característica.

        \item En el sistema de archivos, es muy poco práctica y eficiente 
        la obtención, consulta y modificación de los datos a medida que 
        aumenta la cantidad de información; mientras que las bases de datos 
        tienen mecanismos para realizar consultas y extraer la información 
        como ésta sea requerida y de manera rápida.

        \item En el sistema de archivos es difícil evitar la repetición de 
        información, mientras que las bases de datos nos aseguran que esta 
        repetición sea la mínima posible, lo cual implica menor consumo en 
        disco duro.

        \item En el sistema de archivos el acceso a nuestra información es 
        poco restringido, por lo que se puede robar y copiar la información
        facilmente; mientras que en las bases de datos se pueden definir los 
        usuarios que tienen autorización para acceder a la información o 
        incluso determinar que porción de información puede ver cada tipo de 
        usuario.

        \item En el sistema de archivos puede existir fácilmente la 
        inconsistencia de datos; mientras que las bases de datos deben 
        cumplir con ciertas restricciones de consistencia (la información 
        no se puede perder o sufrir cambios que le hagan perder coherencia).
    \end{itemize}
    
\end{enumerate}
\end{document}