\documentclass[letterpaper,12pt]{article}
% Soporte para los acentos.
\usepackage[utf8]{inputenc}
\usepackage[T1]{fontenc}
% Idioma español.
\usepackage[spanish,mexico, es-tabla]{babel}
% Soporte de símbolos adicionales (matemáticas)
\usepackage{multirow}
\usepackage{amsmath}
\usepackage{amssymb}
\usepackage{amsthm}
\usepackage{amsfonts}
\usepackage{latexsym}
\usepackage{enumerate}
\usepackage{ragged2e}
\usepackage{mathtools}

% Soporte para referencias y citas
\usepackage{hyperref}

% Soporte para imágenes.
\usepackage{graphicx}
% Modificamos los márgenes del documento.
\usepackage[lmargin=2cm,rmargin=2cm,top=2cm,bottom=2cm]{geometry}


    \title{Fundamentos de Bases de Datos \\
        Práctica 02 \\
        Manipulación de archivos}


    \author{Teresa Becerril Torres
            $\#$ de cuenta: $315045132$ \\
            Miguel Ángel Torres Sánchez
            $\#$ de cuenta: $315300442$ \\
            Nicole Romina Traschikoff García
            $\#$ de cuenta: $315164482$ \\
            Tania Michelle Rubí Rojas
            $\#$ de cuenta: $315121719$}
        \date{02 de Septiembre del 2019}

    \begin{document}
    \maketitle

    \section{Implementación en Java.}
    \section{Análisis de requerimientos.}
	Se necesita crear un prototipo de un sistema que capture la información del personal del transporte de la Ciudad de México. La información capturada se administrará en archvivos .cvs y será utilizada posteriormente para poblar una base de datos. El sistema debe:
	\begin{itemize}
		\item Capturar información.
		\item Consultar información.
		\item Editar información.
		\item Eliminar información.
		\item Guardar y leer información en archivos .cvs .
	\end{itemize}

	Los actores del sistema son los empleados, los vehículos, las licencias y los exámenes médicos.\\ \\
	Un empleado contiene su información básica: nombre completo, fecha de nacimiento, dirección, sexo, grado máximo de estudios y el horario en el que labora. Cada empleado conduce un vehículo, tiene una licencia de conducir válida, así como un registro de sus licencias pasadas vencidas, y un examen médico hecho hace no más de 6 meses.
	\\ \\
	Los vehículos tienen un identificador único, saben el tipo de combustible que necesitan y su capacidad de pasajeros. Un vehículo es operado por los empleados y puede ser operado por más de uno.
	\\ \\
	Una licencia contiene un identificador único, su fecha de expedición, su fecha de vencimiento y su tipo de licencia. Las licencias pertenecen a un solo empleado y pueden ser renovadas cuando llegan a su fecha de vencimiento.
	\\ \\
	El examen médico pertenece a un solo empleado y contiene información sobre su peso, talla, presión y estatus. Además mantiene un registro de fecha y hora en que se realizó el examen al empleado y la cédula del médico encargado.
	\\ \\

    \section{Actividad.}
      \begin{enumerate}
        \item Incluir cinco diferencias entre almacenar la información utilizando un sistema de archivos a almacenarla utilizando una base de datos. Describe que es más conveniente utilizar.
      \end{enumerate}


            \end{document}
